\documentstyle[a4]{artikel1}                                                    
                                                                                
\begin{document}                                                                
\title{De Nederlandse Briefstijl}                                               
\author{Victor Eijkhout}                                                        
\maketitle                                                                      
                                                                                
\begin{abstract}                                                                
Een bescheiden handleiding voor de Nederlandse briefstijl,                      
met opmerkingen over het wat, hoe en waarom.                                    
\end{abstract}                                                                  
                                                                                
\section{Verantwoording}                                                        
                                                                                
De Nederlandse \LaTeX-stijl `brief' conformeert zich aan                        
NEN-normen\footnote{Mijn zeer grote dank aan Jan Grootenhuis                    
die mijn aandacht vestigde op het bestaan van de normbladen.}                   
1026 voor briefpapier, 3162 voor het indelen van documenten,                    
1025 voor enveloppen, en 3516 voor het ontwerp van formulieren.                 
                                                                                
Briefontwerp is controversieel. Iedereen heeft een smaak, en                    
met name bij briefpapier is die vaak zeer uitgesproken.                         
Ik wil beklemtonen dat ik bij de implementatie van de briefstijl                
bijna nergens mijn smaak nodig heb gehad. Dankzij de NEN-normen                 
is het ontwerp van briefpapier een zaak van exacte wetenschap,                  
niet van schone kunsten.                                                        
                                                                                
Deze stijl is redelijk incompatibel met de \LaTeX\ `letter' stijl.              
Hij is in bescheiden mate met opties in te stellen,                             
en de sleutelwoorden kunnen uit een aantal talen                                
gekozen worden. Alle teksten zijn verder geparametriseerd.                      
Het is dus goed mogelijk stijlopties te maken om deze stijl                     
aan een specifieke omgeving aan te passen.                                      
                                                                                
Als de gebruiker geen gebruik maakt van een voorgedrukt briefhoofd,             
kan hij zijn eigen briefhoofd in \LaTeX\ implementeren, of                      
door de stijl een briefhoofd geleverd krijgen. Het ontwerp van                  
dit briefhoofd is de enige plaats waar mijn smaak zich heeft                    
doen gelden, maar zelfs dit ontwerp heb ik gejat uit een NEN-norm.              
                                                                                
Opmerking: zeker bij het gebruik van vensterenveloppen is het                   
voor de briefstijl                                                              
van cruciaal belang dat de gebruikte printer goed afgesteld                     
staat. Dit valt te controleren aan de hand van het adres:                       
dit dient op  33mm van de linker kantlijn te staan, terwijl                     
de `baseline' van de eerste regel zich 59mm onder de bovenrand                  
van het papier dient te bevinden.                                               
                                                                                
\section{Velden van het briefpapier}                                            
                                                                                
Deze sectie behandelt de indeling van brieven aan de hand                       
van NEN-norm~1026.                                                              
                                                                                
\subsection{Marges}                                                             
                                                                                
Alle marges conformeren zich aan NEN-1026, behalve de                           
rechtermarge die ik voor een evenwichtiger bladverdeling                        
gelijk gemaakt heb aan de linkermarge.                                          
Met een gezette brief ziet dit er beduidend beter uit.                          
                                                                                
De linkermarge is 33 millimeter; er zijn nog geen voorzieningen                 
voor de versmalde marge van 20 millimeter, die voor facturen                    
beter is.                                                                       
Dit zal misschien ooit een optie worden.                                        
                                                                                
Om aan te sluiten by traditionele getypte briefopmaak                           
is `raggedright' bij verstek ingeschakeld.                                      
Aangezien briefhoofd, referentieregels, en voetregel de brief                   
vrij breed kunnen maken, is de corpsgrootte bij verstek 11-punt.                
10~en~12 zijn opties.                                                           
                                                                                
\subsection{Briefhoofd}                                                         
                                                                                
Het moeilijkste deel van een brief, en de voornaamste plek waar                 
de ontwerper van briefpapier zijn creativiteit kwijt kan                        
is het briefhoofd.                                                              
In de huidige stijl heeft de gebruiker drie mogelijkheden.                      
                                                                                
Een. Hij gebruikt voorgedrukt briefpapier. Als dit zich een                     
beetje aan NEN-normen houdt is er verder niets aan de hand.                     
                                                                                
Twee. De gebruiker kan zelf een macro \verb.\briefhoofd. schrijven.             
Als dit binnen de hoogte van                                                    
\verb.\@headheight. blijft, is er niets aan de hand;                            
voorkeursbreedte is \verb.4\refveldbreedte., dat wil zeggen, de                 
breedte van de referentieregel.                                                 
                                                                                
Drie. Er is een macro van twee argumenten, `maakbriefhoofd',                    
die een briefhoofd levert dat ge\"\i nspireerd is op de                         
voorbeeldbrief in NEN-1026. Voorbeeld:                                          
\begin{verbatim}                                                                
\maakbriefhoofd{WG13}{Werkgroep 13\\ Mathematisch Instituut                     
   \\ Toernooiveld 5\\ 6525 ED Nijmegen}\end{verbatim}                          
Het eerste argument levert                                                      
een tekst op die links boven een verticale lijn gezet wordt.                    
Het links aanlijnen gebeurt met \verb.\hfil., dus met een                       
\verb.\hfill. in het argument kan een deel van het hoofd naar                   
rechts worden geschoven.                                                        
                                                                                
Het tweede argument van `maakbriefhoofd' wordt als een blokje tekst             
rechts onder de lijn in het briefhoofd gehangen.                                
In principe is dit argument er voor het adres van de afzender.                  
Er is echter een mogelijkheid  het antwoordadres                                
in het adresveld op te nemen, zie hier onder.                                   
In het briefhoofd kan dan een omschrijving als                                  
`adviesburo voor gespecialiseerde algemeenheden' staan.                         
                                                                                
Het tekstblok in het tweede argument lijnt links met de datum en                
het vierde voetitem (zie onder), en blijft idealiter binnen                     
de marge van de pagina. Er worden echter geen                                   
`overfull box' meldingen gegeven als de tekst te breed is.                      
                                                                                
Er is een `sterretje'-versie van `maakbriefhoofd';                              
deze geeft op de vervolgbladen alleen het eerste argument                       
plus de streep. Dat maakt de kop van de pagina wat minder                       
zwaar.                                                                          
                                                                                
\subsection{Adresveld}                                                          
                                                                                
Het adresveld wordt zodanig geplaatst dat het in het                            
venster van een vensterenvelop zichtbaar is als het                             
venster 5cm onder de bovenrand van de envelop begint.                           
Omdat er zowel vensterenveloppen zijn met het venster                           
links als rechts, is er een optie {\tt adresrechts}                             
die het adresveld rechts plaatst. De voorkeurspositie                           
is echter links; er blijft dan namelijk rechts een                              
`ontvangerruimte' waar de geadresseerde stempels (`binnen                       
gekomen dd.') en dergelijke kan zetten.                                         
                                                                                
Het adresveld komt op dezelfde manier tot stand als in de                       
oude `letter' stijl: de gebruiker geeft                                         
\begin{verbatim}                                                                
\begin{brief}{Jan \TeX er\\ Overfullplein 10000 \\ Baselinestad}                
\end{verbatim}                                                                  
in, en hieruit destilleert \LaTeX\ de naam en het verdere                       
adres van de geadresseerde.                                                     
De naam komt nog voor op de vervolgbladen.                                      
                                                                                
Het schijnt van de PTT te mogen dat er (helemaal boven) in het ruitje           
van de vensterenvelop een antwoordadres wordt opgenomen,                        
als dat maar gebeurt zodanig dat het geen verwarring                            
schept met het adres. Er is een commando \verb.\antwoordadres.                  
dat een adres als argument accepteert,                                          
met de regels gescheiden door~\verb.\\..                                        
Voorbeeld:                                                                      
\begin{verbatim}                                                                
\maakbriefhoofd{WG13}{De de facto standaard \\ in Vaderlandse                   
                      \\ \TeX verwerking}                                       
\antwoordadres{Mathematisch Instituut \\ Toernooiveld 5                         
               \\ 6525 ED Nijmegen}\end{verbatim}                               
                                                                                
                                                                                
\subsection{De Referentieregels}                                                
                                                                                
Direct onder het adresveld volgt ruimte voor gegevens                           
van de geadresseerde, `Uw brief van' en `Uw kenmerk',                           
en van afzender, `Ons kenmerk' en `Datum'.                                      
                                                                                
Voor deze referentiegegevens staan de gebruiker                                 
de commando's \verb.\uwbriefvan., \verb.\uwkenmerk.,                            
\verb.\onskenmerk., en \verb.\datum. ter beschikking.                           
Deze macro's hebben elk \'e\'en argument tussen accolades.                      
De datum wordt automatisch ingevuld; niet ingevulde                             
gegevens worden niet opgenomen.                                                 
                                                                                
De teksten die hier in de kopjes staan worden                                   
in eerste instantie door de taalopties bepaald.                                 
Verstekwaarde is natuurlijk `nederlands', maar er                               
zijn `engels', `amerikaans', en `duits' beschikbaar.                            
Ik houd me aanbevolen voor correcte terminologie                                
in andere talen; enkele fouten in de Duitse termen                              
zijn op aanwijzing van Marc van Woerkom verbeterd.                              
Wie geeft me de Franse termen?                                                  
                                                                                
De teksten in de referentieregel                                                
zijn waardes van de macros                                                      
\verb.\uwbriefvantekst., \verb.\uwkenmerktekst.,                                
\verb.\onskenmerktekst., en \verb.\datumtekst..                                 
Het is dus mogelijk  `doorkiesnummer' in plaats van `ons kenmerk'               
te krijgen door deze tekst                                                      
als `onskernmerktekst' te declareren, bij voorbeeld                             
met \verb.\renewcommand. in het preamble.                                       
                                                                                
                                                                                
\subsection{De tekst van de brief}                                              
                                                                                
Hier hoeft op deze plaats niets over gezegd te worden,                          
anders dan dat de aanhef, net als in de `letter' stijl                          
met \verb.\opening. gebeurt. Er is een `betreft' commando.                      
                                                                                
Ook de afsluiting staat in het \LaTeX-boek beschreven,                          
maar de commandonamen zijn Nederlands geworden:                                 
`ondertekening' (was `signature'),                                              
`afsluiting' (was `closing'), `ps', en `cc' Dit is Engels,                      
maar bijna niemand weet wat het betekent.                                       
Verder is er zowel een `bijlage' als                                            
`bijlagen' (was~`encl').                                                        
                                                                                
                                                                                
\subsection{Voetruimte}                                                         
                                                                                
`In de voetruimte' (ik citeer NEN-1026) `komen die gegevens                     
van de afzender die niet reeds in het briefhoofd zijn                           
vermeld'. De keuze hiervan wordt aan de gebruiker overgelaten.                  
Enkele suggesties zijn: kantooradres, telefoonnummer van de                     
centrale, faxnummer, inschrijvingsnummer en -plaats in het                      
handelsregister.                                                                
                                                                                
Er kunnen maximaal vier voetitems zijn.                                         
Items verschijnen op de pagina in de volgorder waarin de                        
gebruiker ze gedeclareerd heeft.                                                
Bij twee items of meer is het laatste rechtsgeplaatst. Goeie truc.              
                                                                                
Omdat hier beduidend meer keuze is dan bij de referentieregel,                  
moet de gebruiker zelf het hoofdje en de daaronder                              
geplaatste tekst van items in de voetregel voorschrijven.                       
De macro van twee argumenten \verb.\voetitem. staat                             
hem/haar daartoe ter beschikking. Elke van de argumenten                        
kan meer dan \'e\'en regel lang zijn, gebruik~\verb.\\..                        
Voorbeeld:                                                                      
\begin{verbatim}                                                                
\voetitem{fax:}{12345 Winat nl}                                                 
\voetitem{telefoon:}{080-613169\\ bgg: 612986}                                  
\voetitem{telefoon\\ priv\'e:}{080-448664}\end{verbatim}                        
                                                                                
\subsection{Hulplijntjes; vouwstreepjes}                                        
                                                                                
Door een optie `streepjes' in te schakelen, is het briefpapier                  
van hulpstreepjes te voorzien.                                                  
                                                                                
Volgens NEN-1026 heeft briefpapier een instelstreepje halverwege,               
niet voor vouwen in twee\"en zoals hele volksstammen denken, maar               
voor het aanlijnen van de perforator.                                           
Aan de rechterzijde van het blad zijn twee vouwstreepjes, een                   
voor vouwen in drie\"en, en een voor vouwen in twee\"en.                        
Geen van beide bevindt zich op de helft of een-derde van de                     
papierhoogte. Omdat enveloppen hoger zijn dan een A4-tje gedeeld                
door 2 of~3 zou dat ook niet kunnen. Vensters, weet U wel?                      
                                                                                
Ik ben uitgegaan van envelopformaten EN-C5 ($162\times229$mm                    
voor een A4 in twee\"en) en EN-DL ($110\times220$ voor                          
een A4 in drie\"en) volgens NEN-1025 en ISO 269-1979.                           
Vouwstreepjes bevinden zich op 105mm en 155mm onder de                          
bovenrand van het papier.                                                       
                                                                                
                                                                                
\section{Vervolgbladen}                                                         
                                                                                
Boven elk vervolgblad komt het briefhoofd (mogelijk verkort,                    
zie boven) en een verkorte                                                      
vorm van de referentieregel; deze regel bevat het                               
bladnummer.                                                                     
                                                                                
Als het briefhoofd met daaronder de `vervolgreferentieregel'                    
(galgje! galgje!) te veel ruimte innemen, gaat de extra ruimte                  
af van de teksthoogte. Voor dit mechanisme durf ik niet                         
voor de volle achtien-en-een-half procent in te staan.                          
Het lijkt in ieder geval te werken.                                             
                                                                                
                                                                                
\section{Opties}                                                                
                                                                                
De optie `adresrechts' is  boven al genoemd, evenals `streepjes';               
verder is er nog een optie `typhulp' die een                                    
hulplijntje voor het aanlijnen van het adres zet. Misschien leuk                
als je deze stijl gebruikt voor briefpapier dat daarna voor de                  
typmachine gebruikt wordt.                                                      
                                                                                
Belangrijker zijn natuurlijk de taalopties.                                     
                                                                                
Voor mensen met te goede of slechte ogen zijn er de 10- en                      
12-punts opties. Let op: de kopjes in de referentie- en                         
voetregel blijven in {\tt cmssq8} staan.                                        
                                                                                
Puur voor het genoegen van de implementator van de briefstijl                   
is er de optie `USletter' die het papier 19.7 millimeter                        
korter maakt.                                                                   
                                                                                
\section{Labels}                                                                
                                                                                
Pijnlijk punt. Dit moet nog gedaan worden.                                      
\end{document}                                                                  
                                                                                
